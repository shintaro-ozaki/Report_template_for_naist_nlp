% 更新日2024年1月19日 by Tetsuaki Baba
\documentclass[a4paper,uplatex]{jsarticle}
% 学部は \documentclass[a4paper,uplatex]{jsarticle}を利用して,\section で論文を構成してください.
% 修士、博士は \documentclass[a4paper,uplatex]{jsreport} を利用して, \chapter で論文を構成してください.

\usepackage{iapaper}
\usepackage[dvipdfmx]{graphicx,xcolor}
\usepackage{listings}


\begin{document}
\pagenumbering{arabic} 
\include{code_listings.tex}

\title{レポート名}
\vspace{350pt}
\Yourname{尾崎慎太郎}
\IDnumber{学生番号 2411066}
\submissiondate{2024年1月31日}
\maketitle

\setcounter{page}{1} % 1から振り直す

\report{課題の概要}

\section{問題に関して}

\section{問題に関して}



% 箇条書き
% \begin{itemize}
% \item hoge
% \item foo
% \item bar
% \end{itemize}

% 箇条書き(番号付き)
% \begin{enumerate}
%     \item hoge
%     \item fuga
%     \item bar
% \end{enumerate}

% ソースコードの挿入
% 詳細: https://www.overleaf.com/learn/latex/Code_listing
% \begin{lstlisting}[language=Javascript, caption=p5.js start sample, label={list:label}]
% function setup() {
%   createCanvas(400, 400);
% }

% function draw() {
%   background(220);
% }
% \end{lstlisting}

% \subsection{図表の入れ方}
% 図\ref{fig:tmu_hino}のようにして記述すればok
% \begin{figure}[t]
%   \begin{center}
%     \includegraphics[width=0.95\hsize]{./images/sample.jpg}
%     \caption{東京都立大学日野キャンパス}
%     \label{fig:tmu_hino}
%   \end{center}
% \end{figure}


%%% 参考文献をbibtex形式で引用する場合は上記の参考文献箇所はコメントアウトし、以下をコメントインする。
%%% junsrt: 引用順に文献番号を振る
%%% jplain: アルファベット順に文献番号を振る
\bibliographystyle{junsrt}
\bibliography{references}
\end{document}
